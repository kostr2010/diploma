\chapter{Обзор существующих решений и выводы}
\label{sec:Chapter2} \index{Chapter2}

Необходимость поддержки множества анализаторов обуславливается еще и тем, что каждый анализатор покрывает лишь конечное число уязвимостей, упуская остальные\cite{Delaitre2013OfMS} \cite{3291} \cite{Bessey2010AFB}. Поэтому для получения наиболее полного по типам ошибок датасета необходимо уметь обрабатывать сообщения от разных анализаторов. Решением данной проблемы является приведение всех кодов ошибок к кодам CWE \cite{CWE-doc} на этапе обработки сообщения и использование соответствующего кода CWE в датасете при обучении и классификации. Таким образом, добавление поддержки нового анализатора заключается в трансляции кода ошибок анализатора в CWE. Более подробно данный метод будет описан в дальнейшем в соответствующем разделе.

Sarah Heckman и Laurie Williams в своем исследовании\cite{HECKMAN2011363} выделили следующие характеристики, используемые в наиболее релевантных работах о классификаторах, кодобных тому, который является целью данного диплома:
\begin{enumerate}
    \item Характеристики ошибки - атрибуты предупреждения, сгенерированного статистиким анализатором, например: тип ошибки (double free, etc.), местоположение в коде (файл, класс, функция, строка, etc.), а также приоритет, который анализатор присвоил ошибке
    \item Характеристики кода - метрики кода, в котором содержится предупреждение. Эти метрики могут быть извлечены при помощи дополнительного анализа или из самого кода (цикломатическая сложность, количество строк в файле, etc.)
    \item Метрики репозитория с исходным кодом - атрибуты репозитория с исходным кодом (история коммитов, частота изменения кода, история ревью, etc.)
    \item Метрики базы данных с багами - информация о багах может быть связана с изменениями в исходном коде для того, чтобы идентифицировать уязвимость и необходимые исправления
    \item Метрики динамического анализа - гибридное использование статического и динамического анализа может помочь митигировать затраты, связанные с исполнением каждой техники анализа
\end{enumerate}

В данной работе внимание было сосредоточено целиком на первых двух пунктах: характеристики ошибки и характеристики кода. Изучив соответствующие статьи, было выяснено, что основными характеристиками кода для последующего анализа являются метрики, такие как число вложенности фукции, число условных переходов, цикломатическая сложность, etc \cite{test-suites-for-dataset}. Однако все эти метрики являются метаданными, которые лишь характеризуют код в целом, не давая понятия о его структуре. Для того, чтобы каким либо образом анализировать структуру кода обычно используется представление AST\cite{Shedko2020ApplyingPM}. Такой подход хорош для таких языков как Java, Python, где AST может быть легко получено. Для получения же AST в C/C++ необходина полная компиляция проекта. Т.к. каждый проект имеет свои систему для сборки, то невозможно описать данную процедуру единым простым путем. Поэтому в данной работе был опробован подход на основе token-based представления кода, предложенного в \cite{Shedko2020ApplyingPM} для задач определения стиля проекта и вывода правил использования API. Идея состоит в том, что хоть представление в виде токенов и является неточным, но отфильтровав его, и анализируя наряду с метаданными кода, может быть получена более полная картина, нежели чем при использовании только метаданных. Более подробно этот подход будет описан в дальнейшем в соответствующем разделе.

Juliet Test Suite - это набор рукописных тестов, предназначенных для валидации эффективности различных статистиких анализаторов\cite{Juliet}. Ugur Koc с коллегами уже использовали эту тестовую сюиту для обучения своих моделей классификации. Также данный метод был использован Lori Flynn с коллегами \cite{test-suites-for-dataset}. Подход к генерации датасета, используемый в данной работе был позаимствован из этих работ.

\newpage