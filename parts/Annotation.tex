\begin{abstract}

    \begin{center}
        \large{Применение методов машинного обучения в статистическом анализе} \\
        \large\textit{Назаров Константин Олегович} \\[1 cm]
    \end{center}


    Статические анализаторы широко используются в промышленности для нахождения и исправления уязвимостей в коде в процессе разработки, тем самым улучшая качество кода и упрощая дальнейшую разработку. Однако их применение ограничивается тем, что на больших и сложных кодовых базах статические анализаторы выдают большое количество false positives наряду с предупреждениями о действительных уязвимостях. Поэтому значительная часть данной работы посвящена тому, чтобы применить методы машинного обучения для классификации и приоритизации сообщений статического анализатора, что позволит сократить время, затрачиваемое разработчиками на ручную обработку всех сообщений анализатора. Основной проблемой при обучении подобного классификатора является составление тренировочного датасета, т.к. неочевиден алгоритм разметки данных и признаки для обучающей выборки. Также большим фактором является скорость генерации: признаки не должны вычисляться долго, т.к. сам статический анализ занимает очень большое время (для относительно простых анализаторов время работы примерно в 2 раза превышает время компиляции, но может быть и много больше \cite{GCC-SA}). В данной работе рассматривается способ генерации датасета при помощи тестовых сюит для статических анализаторов (т.е. репозиториев с "образцовыми" программами, специально написанными для тестирования статических анализаторов), а также показывается, что включение в обучающий датасет token-based представления кода помогает увеличить способность к распознаванию разных шаблонов ошибок, не увеличивая при этом сложности генерации датасета. Данные для обучения были получены путем применения нескольких статических анализаторов к кодовой базе Juliet C/C++ Test Suite v1.3. Разметка была произведена автоматически из метаданных тестовой сюиты. Полученный датасет был использован для обучения двух моделей: decision tree и gradient boosting, предоставляемых библиотекой xgboost для python3. Полученные классификаторы имеют высокую точность на тестовой выборке ($\approx 96~\%$). Также в работе был предложен альтернативный метод генерации датасета, призванный решить проблемы использованного метода. Применение альтернативного метода является предметом дальнейших исследований.

    \vfill

\end{abstract}
