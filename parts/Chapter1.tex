\chapter{Постановка задачи}
\label{sec:Chapter1} \index{Chapter1}

Основной целью данной работы является сокращение времени, затрачиваемого разработчиком на обработку результатов статического анализа. Также важной целью является применение полученных результатов для анализа работы статического анализатора и его дальнейшей настройки. Для достижения обозначенной цели требуется решить следующие задачи:

\begin{enumerate}
    \item Определить признаки для будущего датасета
    \item Определить метод генерации датасета и его разметки. Собрать данные
    \item Обучить классификатор сообщений статического анализатора
    \item Проанализировать полученные результаты
\end{enumerate}

Для того, чтобы результат работы считался удовлетворительным и пременимым на больших масштабах, решения для поставленных задачи должны удовлетворять следующим требованиям:

\begin{enumerate}
    \item Сложность извлечения признаков не должна превосходить сложности статического анализа
    \item Разметка датасета должна быть автоматизированной, т.к. ручная разметка занимает слишком долгое время и требует наличия проанализированной большой кодовой базы \cite{Ayewah2010TheGF}
    \item Классификатор не должен быть ограничен конкретными анализаторами, т.е. должен иметь возможность обработать и классифицировать предупреждение от любого анализатора после необходмой обработки
\end{enumerate}

\newpage