\section{Заключение}
\label{sec:Chapter5} \index{Chapter5}

В результате данной работы:
\begin{itemize}
    \item Был создан классификатор сообщений предупреждений статического анализатора на два класса: true positive и false positive. Полученный классификатор позволяет приоритизировать предупреждения, т.к. помимо предсказания может указывать еще и уверенность в нем. Классификатор показал высокий precision (95 и 98\% для decision tree и GBDT соответственно) и приемлемый recall (88 и 94\%) на тестовой выборке. Также, классификатор на основе модели градиентного бустинга показал лучшую обобщаюцую способность, корректно отработав на представленных реальных примерах. На большую гибкость модели градиентного бустинга в данной задаче указывает еще и высокие (по сравнению с решающим деревом) precision и recall.
    \item Был рассмотрен метод генерации датасета при помощи тестовых сюит для статических анализаторов. Рассмотренный метод был реализован и использован для обучения вышеописанных моделей. Удалось найти набор признаков, время на извлечение которых значительно быстрее времени компиляции проекта и не требует его сборки. Это является большим плюсом, т.к. для языков C/C++ процесс сборки может быть сложным и не является стандартизованным. Поэтому возможность извлекать признаки из отдельных файлов очень полезна и заметно упрощает генерацию датасета на произвольной кодовой базе.
    \item Было обсуждено, как полученные в результате обучения модели могут быть полезны при разработке и поддержке статических анализаторов разработчиками.
\end{itemize}

В ходе работы большинство поставленных задач было реализовано. Также был обозначен фронт будущих работ, нацеленных на улучшение полученного результата и исправление всех обозначенных в процессе работы минусов.

\newpage