\section{Введение}
\label{sec:Chapter0} \index{Chapter0}

\subsection{Обозначения и сокращения}

\begin{enumerate}
    \item False Positive, FP - ошибка бинарной классификации, при которой классификатор ошибочно предсказывает положительный результат (например диагноз болен, когда болезни на самом деле нет).
    \item True Positive, TP - случай, когда бинарный классификатор верно предсказал положительный результат (диагноз болен, когда болезнь есть)
    \item Датасет - это таблица, строки которой называются объектами, а столбцы – признаками этих объектов. Иногда датасет включает в себя еще и метку для объектов.
    \item Token-Based представление - представление кода в виде последовательности токенов, получаемых в результате лексического анализа.
    \item Тестовая сюита - специализированный набор тестов.
    \item Abstract Syntax Tree, AST - представлнение исходного кода в виде дерева абстрактного синтаксиса.
    \item Application Programming Interface, API - описание способов, которыми одна компьютерная программа может взаимодействовать с другой программой.
    \item Common Weakness Enumeration, CWE \cite{CWE-doc} - Список уязвимостей программного и аппаратного обеспечения. Является общепринятым языком для описания природы уязвимости.
    \item NIST - The National Institute of Standards and Technology.
    \item SATE - Static Analysis Tool Expositions.
\end{enumerate}

\subsection{Введение}

Статические анализаторы широко используются в промышленности для нахождения и исправления уязвимостей в коде в процессе разработки, тем самым улучшая качество кода и упрощая дальнейшую разработку. Однако их применение ограничивается Множеством факторов. Статья "A Few Billion Lines of Code Later. Using Static Analysis to Find Bugs in the Real World"\cite{Bessey2010AFB} раскрывает множество проблем с применением статических анализаторов на практике, среди которых: большое количество false positives ($35\% - 91\%$ \cite{HECKMAN2011363})  на сложных или больших кодовых базах; пользователи, помечающие непонятные им сообщения как ложные; а также сложность обработки большого количества ошибок пользователем, которая приводит к тому, что анализаторы выдают меньше сообщений, чем могли бы. В связи с этим, Moritz Marc Beller с коллегами \cite{Beller2016AnalyzingTS} обнаружили, что на практике очень малое число проектов с открытыми исходным кодом интегрирует статические анализаторы в процесс разработки. Также большая часть проектов не поддерживается правила, что предупреждения компиляции должны рассматриваться как ошибки.  Поэтому значительная часть данной работы посвящена тому, чтобы применить методы машинного обучения для классификации и приоритизации сообщений статического анализатора, что позволит сократить время, затрачиваемое разработчиками на ручную обработку всех сообщений анализатора а также облегчит эффективное использование анализаторов на практике. Основной проблемой при обучении подобного классификатора является составление тренировочного датасета, т.к. неочевиден алгоритм разметки данных и признаки для обучающей выборки. Также большим фактором является скорость генерации: признаки не должны вычисляться долго, т.к. сам статический анализ занимает очень большое время (для относительно простых анализаторов время работы примерно в 2 раза превышает время компиляции, но может быть и много больше \cite{GCC-SA}). В промышленности эта проблема решается тем, что датасет зачастую имеется\cite{Ruthruff2018PredictingAA} \cite{classification-models-multiple-SA-tools} или может быть легко получен из статистики использования или из истории проекта\cite{assesing-validity-of-sa-warnings-cisco}. В данной работе рассматривается способ генерации датасета при помощи тестовых сюит для статических анализаторов\cite{test-suites-for-dataset} (т.е. репозиториев с "образцовыми" программами, специально написанными для тестирования статических анализаторов), а также показывается, что включение в обучающий датасет token-based представления кода помогает увеличить способность к распознаванию разных шаблонов ошибок, не увеличивая при этом сложности генерации датасета. Данные для обучения были получены путем применения нескольких статических анализаторов к кодовой базе Juliet C/C++ Test Suite \cite{Juliet}. Разметка была произведена автоматически из метаданных тестовой сюиты. Полученный датасет был использован для обучения двух моделей, которые впоследствии были протестированы на тестовой выборке.

\newpage